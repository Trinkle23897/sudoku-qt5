\documentclass[a4paper]{article}
\usepackage{latexsym,amssymb,amsmath,amsbsy,amsopn,amstext,xcolor,multicol}
\usepackage{ctex,hyperref,graphicx,wrapfig,fancybox,listings}
\usepackage{pgf,pgfarrows,pgfnodes,pgfautomata,pgfheaps,pgfshade}
\usepackage[top=1in, bottom=1in, left=1.25in, right=1.25in]{geometry}
\graphicspath{{pic/}}
\title{\includegraphics[scale=0.05]{sudoku.png} \bf Sudoku Game}
\date{2017.9}
\author{计64~~翁家翌~~2016011446}
\begin{document}
\kaishu
\ttfamily
\maketitle
%\tableofcontents
%\newpage
\section{软件用途}
本软件是一个数独小游戏,使用Qt5编写,实现了数独游戏的快速生成、以及提供任意题目的解决方案。
\section{运行方式}
安装Qtcreater之后,将源代码拷贝至本机,源代码位于\url{https://git.thusaac.org/trinkle/sudoku-qt5}。运行Qtcreater直接编译即可。

Ubuntu下需要额外安装软件包qtmultimedia5-dev,进入目录之后运行 \uline{qmake -makefile;make;} 即可得到可执行文件 sudoku。
\section{功能介绍}
\subsection{使用帮助}
\begin{figure}[htp]
\centering
\includegraphics[width=0.5\linewidth]{help.png}
\caption{提示信息}
\label{fig:help}
\end{figure}

图~\ref{fig:help} 显示了关于本软件的使用帮助。它是运行软件时的最初界面,点击\uline{OK}即可进入主游戏界面。
\subsection{游戏界面}

\begin{figure}[htp]
\centering
\includegraphics[width=0.7\linewidth]{start.png}
\caption{开始界面}
\label{fig:start}
\end{figure}

图~\ref{fig:start} 显示了软件的开始游戏界面。最上方为选项栏,左侧是菜单栏,中部是数独界面,右上侧是功能栏,右中部是计时器,右下侧是数字栏。

\subsection{选项栏}
\begin{figure}[htp]
\centering
\includegraphics[width=0.5\linewidth]{title.png}
\caption{标题选项栏}
\label{fig:title}
\end{figure}

图~\ref{fig:title} 显示了软件的标题选项栏,有四种难度可供选择,每种难度下有内置4个关卡,并且还有按照该难度随机生成题目的选项。

最后一个选项Custom提供了输入题目的功能,该软件能够给出用户提供题目的解答。
\subsection{菜单栏}
\begin{figure}[htp]
\centering
\includegraphics[width=0.05\linewidth]{menubar.png}
\caption{菜单栏}
\label{fig:menu}
\end{figure}

图~\ref{fig:menu} 显示了软件的左侧菜单栏,从上到下依次为:显示信息(快捷键:\uline{Alt+I})、暂停/继续游戏(快捷键:\uline{空格})、撤销(快捷键:\uline{Ctrl+Z})、重做(快捷键:\uline{Ctrl+Y})、回到初始题目状态、音效开关(快捷键:\uline{Ctrl+M})。
\subsection{功能栏}
\begin{figure}[htp]
\centering
\includegraphics[width=0.3\linewidth]{function.png}
\caption{功能栏}
\label{fig:func}
\end{figure}

图~\ref{fig:func} 显示了软件的右上侧功能栏,从上到下依次为:开始游戏、显示帮助、显示计算机给出的解答。

显示帮助时,软件会显示一个还未被正确填上数字的格子的答案;显示帮助和解答时,均不会覆盖用户记录,下一步操作会回到点击按钮之前的状态。
\subsection{数字栏}
\begin{figure}[htp]
\centering
\includegraphics[width=0.2\linewidth]{number.png}
\caption{数字栏}
\label{fig:num}
\end{figure}

图~\ref{fig:num} 显示了软件的数字栏,点击$1\sim 9$(或者按下数字$1\sim 9$按键)即可在当前选中的格子上标记/去除一个点击的数字。第四行第一个图标为清除一个格子内的所有数字(快捷键:\uline{Delete}),第二个图标为标记/取消标记一个选中的格子(快捷键:\uline{M})。
\subsection{数独界面}
\begin{figure}[htp]
\centering
\includegraphics[width=1\linewidth]{status.png}
\caption{运行界面}
\label{fig:status}
\end{figure}

图~\ref{fig:status} 显示了软件在游戏过程中的运行界面。在每个时刻,均有一个格子被选中,用黑框蓝色字体高亮显示。当选中的格子已经填上一个数字,那么所有相同数字的格子均会立体高亮显示;否则如果选中的是一个空白格子,那么与这个格子同行同列的所有格子都会被立体高亮显示。当前选中格子的坐标在左下角会给予显示。

如果一个格子是题目数据,那么它会被标记为黑色字体;如果一个格子里有多个数字,那么程序将会以灰色小字号在格子内显示这些数字;如果一个填写上去的数字和已有数字发生冲突,则标记为红色字体;否则标记为绿色字体。

\begin{figure}[htp]
\centering
\includegraphics[width=0.7\linewidth]{finish.png}
\caption{完成解题}
\label{fig:finish}
\end{figure}

图~\ref{fig:finish} 显示了软件在用户完成解题时的界面。

\begin{figure}[htp]
\centering
\includegraphics[width=0.7\linewidth]{solve.png}
\caption{自动解题}
\label{fig:solve}
\end{figure}

图~\ref{fig:solve} 显示了软件在Custom模式中按下 \fbox{Solve} 按钮之后的自动解题情况,黑色为用户输入数据,左下角状态栏会显示是否多解/单解/无解,以及计算机解题的用时。
\end{document}